\documentclass[12pt,a4paper]{article}
\usepackage{fullpage}
\usepackage{amsmath, amssymb}

\begin{document}
\begin{flushleft}

\textbf{An Introduction to Cryptography\\}
~\\
\textbf{1.1 Simple substitution ciphers\\}
Cryptanalysis: \\
1.There are $26! \approx 10^{26}$ different simple substitution ciphers, each is regarded as a key.\\
2.The frequency of each letter appears is available.\\
~\\
\textbf{Divisibility and greatest common divisors\\}
\textbf{Theorem}(The Euclidean Algorithm). Let a and b be positive integers with a $\geq$ b. The following algorithm computes gcd(a,b) in a finite number of steps.\\
(1) Let $r_0$ = a and $r_1$ = b.\\
(2) Set i = 1.\\
(3) Divide $r_{i-1}$ by $r_i$ to get a quotient $q_i$ and remainder $r_{i+1}$,\\
\hspace{20mm}$r_{i-1} = r_i * q_i + r_{i+1}$\hspace{5mm}with $0 \leq r_{i+1} \leq r_{i}$.\\
(4) If the remainder $r_{i+1}$ = 0, then $r_i$ = gcd(a,b) and the algorithm terminates.\\
(5) Otherwise, $r_{i+1} \geq 0$, so set i = i + 1 and go to step 3.\\
The division step (3) is executed at most $2log_2(b) + 1$ times.\\
~\\
\textbf{Theorem}(Extended Euclidean Algorithm). Let a and b be positive integers. Then the equation\\
$$au + bv = gcd(a,b)$$
always has a solution in integers u and v.\\
If $(u_0, v_0)$ is any one solution, then every solution has the form\\
$u = u_0 + \frac{b*k}{gcd(a,b)}$ \hspace{2mm} and \hspace{2mm} $v = v_0 - \frac{a*k}{gcd(a,b)}$ \hspace{5mm} for some k $\in \mathbb{Z}$.\\

\end{flushleft}
\end{document}